\documentclass[a4paper]{article}
\usepackage[utf8]{inputenc}
\usepackage[spanish]{babel}
\usepackage{siunitx}
\usepackage[margin=2cm]{geometry}
\usepackage[contents,nonumber]{cuisine}
\usepackage{microtype}
\usepackage[colorlinks=true,linkcolor=blue,filecolor=blue,urlcolor=red,bookmarks=true,]{hyperref}

% Necesario para remover el numerado dentro de las recetas.
\usepackage{xpatch}
\makeatletter
\xpatchcmd{\Displ@ySt@p}{\arabic{st@pnumber}}{}{}{}
\makeatother

% Números en las footnotes
\renewcommand{\thempfootnote}{\arabic{mpfootnote}}

\title{KISS cooking}
\author{Lucas Collino}

\begin{document}

\maketitle

\newpage

\begin{abstract}

\vspace*{3\baselineskip}

Esto es una recopilación de recetas que me vuelan la peluca. Me cansé de mantener papeles garabateados que pierdo todo el tiempo.

Cada receta se puede hacer de cero, sin ningún prerequisito y sin complicaciones innecesarias.

Cada receta la rockea.

La declaración de ingredientes es lo más explícita posible, trato de evitar medidas subjetivas.

Comprate una balanza digital para pesar gramos. Dejate de joder.

La última versión de este recetario se encuentra \href{https://github.com/lucascollino/kiss_cooking}{acá}.

\vspace*{6\baselineskip}

\begin{flushright}
¡¡¡Vamos a opensourcear recetas!!!
\end{flushright}

\end{abstract}

\newpage

\tableofcontents

\newpage

\begin{recipe}{Pizza al molde (porque no existe otra)}{3-4 Pizzas}{\SI{120}{\minute}}
  \ingredient[650]{\SI{}{\cubic\centi\metre}}{Agua}
  \ingredient[50]{g}{Levadura}
  \ingredient[7]{g}{Azúcar}
  \textbf{Masa}

  Medir el agua y entibiarla en el microondas. Desgranar el cubo de levadura en el agua, agregar el azúcar y revolver.

  \ingredient[1]{kg}{Harina}
  \ingredient[20]{g}{Sal}
  \ingredient{}{Aceite de girasol}
  Colocá el kg de harina y la sal en una olla, mezclalo.

  Sumá el agua y mezclá hasta lograr una masa homogénea\footnote{Usá una sola mano.}.

  Sacá la masa de la olla y amasá el tiempo que tengas ganas hasta que esté pareja. Espolvoreá con harina el bollo y metelo en la olla de nuevo, dejalo levar por \SI{45}{\minute}.

  Dividí el bollo en tres\footnote{En cuatro si los moldes son chicos.} y estiralo en los moldes al \SI{50}{\percent}, dejalo descansar unos \SI{10}{\minute} (hacé la salsa), y estiralos al \SI{100}{\percent} del molde. Dejalos levar otros \SI{45}{\minute}.

  \ingredient[1]{lata}{Tomate perita}
  \ingredient[50]{ml}{Aceite de girasol}
  \ingredient[7]{g}{Sal}
  \ingredient[7]{g}{Azúcar}
  \ingredient{}{Orégano}
  \ingredient{}{Pimienta}
  \ingredient{}{Paprika}

  \textbf{Salsa}

  En el mismo recipiente donde mediste el agua, poné el tomate perita, orégano, paprika y pimienta a gusto, junto con la sal, azúcar y el aceite. Licualo con una minipimer.

  \ingredient[1.2]{kg}{Queso cremoso}
  \textbf{Cocción}
  
  Las siguientes referencias son para un horno eléctrico de \SI{250}{\celsius} máximo, con resistencia inferior y superior, sin convección. Es importante que esto lo ajustes a tu horno (puede variar, y mucho).

  Prendé el horno al máximo unos \SI{10}{\minute}.
  
  Repartí la salsa con una cuchara o cucharón grande.
  
  Mandá una pre-pizza -todavía sin el queso- en la bandeja superior del horno \SI{3}{\minute}, luego reemplazala con otra y a la que rotas mandála a la chapa de abajo para que se dore la base de la masa otros \SI{3}{\minute}.
  
  Cuando completan los dos ciclos les pones el queso.
  
  Una vez que ya está liberada la bandeja superior, mandá la primer pizza con el queso y dejala \SI{8}{\minute}, repetí con el resto. La base tiene que estar dorada y el queso fundido, burbujeando y dorándose.

  \vspace*{1\baselineskip}
  \textbf{Pro-tips}

  \begin{enumerate}
    \item[No uses moldes de aluminio] Es imposible que no se pegue la masa, y cuesta que se dore bien. Usa enlozadas o en su defecto teflón.
    \item[Usa queso cremoso] Muzzarella va mejor con pizza a la piedra.
    \item[Ponele salsa en los bordes] La combinación de salsa y queso en los bordes al quemarse es indescriptible.
   \end{enumerate}

\end{recipe}

\end{document}